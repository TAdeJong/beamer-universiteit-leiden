\documentclass[9pt,aspectratio=34]{beamer}

\usetheme[style=fwn]{leidenuniv}
\title{Beamer template for Leiden University}
\subtitle{A minimal example showcasing the options}
\author{Tobias A. de Jong}
\institute[LION]{Leiden Institute of Physics}
\date{\today}
\titlegraphic{
	\includegraphics[height=0.3\paperheight]{logo-universiteitleiden-english.pdf}
}
\begin{document}
\begin{frame}
	\titlepage
\end{frame}

\begin{frame}
	\tableofcontents
\end{frame}
\section{A section}
\subsection{With a subsection}
\begin{frame}
\frametitle{There Is No Largest Prime Number}
\framesubtitle{The proof uses \textit{reductio ad absurdum}.}
\begin{theorem}
There is no largest prime number.
\end{theorem}
\begin{proof}
\begin{enumerate}
\item<1-| alert@1> Suppose $p$ were the largest prime number.
\item<2-> Let $q$ be the product of the first $p$ numbers.
\item<3-> Then $q+1$ is not divisible by any of them.
\item<1-> But $q + 1$ is greater than $1$, thus divisible by some prime
number not in the first $p$ numbers.\qedhere
\end{enumerate}
\end{proof}
\end{frame}

\begin{frame}
	\frametitle{Block colors}
	\begin{block}{A block}
		With text
	\end{block}
	\begin{alertblock}{An alert block}
		With text
	\end{alertblock}
	\begin{exampleblock}{An example block}
		\begin{itemize}
			\item An item
			\item And another one
		\end{itemize}
	\end{exampleblock}
\end{frame}
\end{document}
